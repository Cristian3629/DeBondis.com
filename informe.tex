\documentclass[10pt,a4paper]{article}
\usepackage[latin1]{inputenc}
\usepackage[spanish]{babel}
\usepackage{amsmath}
\usepackage{amsfonts}
\usepackage{amssymb}
\usepackage{graphicx}
\usepackage{vmargin}
\setpapersize{A4}
\setmargins
{2.5cm}       % margen izquierdo
{1.5cm} % margen superior
{16.5cm}% anchura del texto
{23.42cm} % altura del texto
{10pt} % altura de los encabezados
{1cm} % espacio entre el texto y los encabezados
{0pt} % altura del pie de p�gina
{2cm}% espacio entre el texto y el pie de p�gina

\begin{document}
	
	\begin{titlepage}
		\centering
		{\scshape\LARGE Universidad de Buenos Aires \par}
		\vspace{1cm}
		{\scshape\Large Taller de Programaci�n (75.42)\par}
		\vspace{1.5cm}
		{\huge\bfseries DeBondis.com \par}
		\vspace{2cm}
		{\Large\itshape Cristian Gonz�lez\par}
		\vspace{2cm}
		{\Large\itshape 94719 \par}
	\end{titlepage}
	
	
	
	
	
	
	%�ndice.
	\tableofcontents
	\newpage
\section{Clases}
\subsection{Server}
Modela la responsabilidad del servidor, se encarga de subir toda la informaci�n a partir de los archivos, c�mo tambi�n hacer que todos los hilos puedan acceder a toda la informaci�n sobre los recorridos y los colectivos en tr�nsito.

\subsection{Client}
Se encarga de poder interpretar los comandos y enviar la informaci�n pertinente para su consulta, una vez que tenga una respuesta del server (ya sea un error o no) imprimir el resultado por pantalla al cliente.

\subsection{Date}
Dise�e esta clase para encapsula la fecha y hora, a modo de encapsular un poco el comportamiento y poder usarlo con mayor comodidad, por ejemplo haciendo sobrecarga del operator -. Tambi�n tiene una responsabilidad de dado un timeLinux poder setear todas las variables, hora, d�a, etc.


\subsection{WaitCharacter}
Clase que hereda de Thread que encarga de esperar por stdin el caracter "q" y en caso de que suceda eso, avisar el server que debe "apagarse".

\subsection{TimeStop}

Esta se encarga de encapsula el tiempo que existe entre 2 Paradas, de tal forma poder consultarle el tiempo que existe entre las paradas, si cierta parada es el first o el end.



\subsection{WaitClient}

Se encarga de esperar una nueva conexi�n, una vez que un nuevo cliente se conecta, este se encarga de lanzar otro hilo para atenderlo, c�mo tambi�n se encarga de dada un vez conexi�n chequear c�mo se encuentran los otros clientes (si el cliente de cierto hilo tiene alguna nueva consulta). Para la decisi�n de chequear cada cierto tiempo fue a modo de que supongo que el lapso entre que se conecta un cliente y otro no es mucho.

\subsection{Thread}
Se encarga de de modelar el comportamiento de un thread, para crearlo se tiene que proveer los datos que va a necesitar ese thread como la funci�n/m�todo a ejecutar.


\subsection{Colectivo}
Se encarga de ser contener la hora y fecha en la cual sali� y qu� recorrido tiene. 

\subsection{Colectivo Recorrido}
Este encapsula todo el recorrido de un colectivo, es la informaci�n que se sube de paradas.txt y bondis.txt de tal forma que en esta clase se puede obtener toda la informaci�n respecto a tiempo para llegar de una parada a otra o s� pasa por cierta parada.

\subsection{Attend Client}
Esta es la clase m�s compleja, se encarga de resolver las consultas de un cliente, poder pedirle al server toda la informaci�n necesaria para resolver dicha consulta y enviar al cliente la respuesta de la query.



\begin{figure}[h]
\includegraphics[width=1.1\linewidth]{Attend}

\label{fig:Attend}
\end{figure}


\section{Links}
Link del repositorio:https://github.com/Cristian3629/DeBondis.com

\end{document}